\documentclass[a4paper,11pt]{jsarticle}

\usepackage{amsmath,amsfonts}
\usepackage{amssymb}
\usepackage{bm}
\usepackage[dvipdfmx]{graphicx}
\usepackage{ascmac}
\usepackage{fancybox}
\usepackage{tikz}
\usepackage{amsthm}
\usepackage{expl3}
\usepackage{ytableau} 
\usepackage{enumerate}
\usepackage{ulem}
\usepackage{mathtools}
\usetikzlibrary{positioning, intersections, calc, arrows.meta,math, decorations.markings}


\theoremstyle{plain}
\newtheorem{thm}{Theorem}
\newtheorem*{thm*}{Theorem}

\theoremstyle{definition}
\newtheorem{dfn}{Definition}

\newtheorem{lem}{Lemma}

% 括弧のサイズを自動調整
\renewcommand{\(}{\left(}
\renewcommand{\)}{\right)}
\renewcommand{\[}{\left[}
\renewcommand{\]}{\right]}
\renewcommand{\{}{\left\lbrace}
\renewcommand{\}}{\right\rbrace}

\newcommand{\abs}[1]{\left| #1 \right|} % 絶対値

% 矢印
\newcommand{\Iff}{\Longleftrightarrow}
\newcommand{\To}{\Longrightarrow}
\newcommand{\ShortTo}{\Rightarrow}

% 行列
\newcommand{\pmat}[1]{\begin{pmatrix} #1 \end{pmatrix}}
\newcommand{\mat}[1]{\left( \begin{matrix} #1 \end{matrix} \right)}

% 微分
\ExplSyntaxOn
\newcommand{\ppartial}[1]
    {
        \pd_parse:n { #1 }
    }
    \cs_new_protected:Nn \pd_parse:n
    {
    \seq_set_split:Nnn \l_tmpa_seq { , } { #1 }
    \partial \seq_use:Nn \l_tmpa_seq { \partial }
}

\newcommand{\pd}[2]{
    \frac{\partial{#1}}{\ppartial{#2}}
}
\ExplSyntaxOff

% 集合
\newcommand{\R}{\mathbb{R}}
\newcommand{\N}{\mathbb{N}}
\newcommand{\Z}{\mathbb{Z}}
\newcommand{\Q}{\mathbb{Q}}
\newcommand{\C}{\mathbb{C}}
\newcommand{\F}{\mathbb{F}}
\newcommand{\K}{\mathbb{K}}

\begin{document}

\title{2.1. シューア多項式}
\author{dragoemon}
\date{\today}
\maketitle

\ytableausetup{centertableaux}
\ytableausetup{boxsize=1.5em}



\section{タブロー環に慣れよう}

\begin{itembox}[l]{定義}
    $\lambda$を$[m]$上のタブロー全体からなる集合を$\text{Tab}^{[m]}(\lambda)$と書くことにする。$[m]$が明らかな場合は単に$\text{Tab}(\lambda)$と書く。
\end{itembox}

本にはこの記号が使われていないが、便利なので使うことにする。





\begin{itembox}[l]{定義}
    $[m]$上のタブロー全体のモノイド$M_m$は群環$R_{[m]}$を定める。
    \begin{align*}
        R_{[m]} = \{ \sum_{k=1}^{n} a_k T_k | n \in \Z_{\geq 0},  \{a_k\}_{k=1}^{n} \subset \Z, \{T_k\}_{k=1}^{n} \subset M_m\}
    \end{align*}
    
    $[m]$上のヤング図形$\lambda$に対して、$R_{[m]}$の元$S_{\lambda}$を
    \begin{align*}
        S_{\lambda} = \sum_{T \in \text{Tab}^{[m]}(\lambda)} T
    \end{align*}
    で定める。
\end{itembox}

ここでタブローの整数倍、和はあくまで形式的なものであって、計算できるものではない。(群環とはそういうものである)。
集合論的に厳密に構成したいのであれば、タブロー$T$をデルタ関数$\delta_{T}(U) = \begin{cases} 1 & (U = T) \\ 0 & (U \neq T) \end{cases}$と同一視して、
タブローから整数への写像全体がなす加群に、積の演算を入れたものだと思っても良い。
しかし分かりにくいと思うので、具体的な計算を通して理解することにする。


\begin{itemize}
    \item $R_{[3]}$において$S_{(2,1)}$を計算すると、
    \begin{align*}
        S_{(2,1)} &= \ytableausetup{boxsize=0.5em} S_{\ydiagram{2,1}}  = \sum_{T \in \text{Tab}^{[3]}\(\ydiagram{2,1}\)} T \ytableausetup{boxsize=1.5em} \\
        &= \ytableaushort[]{11,2} +  \ytableaushort[]{11,3} + \ytableaushort[]{12,2}  + \ytableaushort[]{12,3} +  \ytableaushort[]{13,3} +  \ytableaushort[]{22,3} +  \ytableaushort[]{23,3}
    \end{align*}
    \item $R_{[2]}$において$\( \ytableaushort[]{1} + \ytableaushort[]{2} \)^2$を計算すると、
    \begin{align*}
        \(\ytableaushort[]{1} + \ytableaushort[]{2}\)^2 &= \ytableaushort[]{1} \cdot \ytableaushort[]{1} + \ytableaushort[]{1} \cdot \ytableaushort[]{2} + \ytableaushort[]{2} \cdot \ytableaushort[]{1} + \ytableaushort[]{2} \cdot \ytableaushort[]{2} \\
        &= \ytableaushort[]{11} + \ytableaushort[]{12} + \ytableaushort[]{1,2} + \ytableaushort[]{22}
    \end{align*}
\end{itemize}

\begin{itembox}[l]{問題}
    $R_{[2]}$において、
    \begin{align*}
        S_{(2)} \cdot S_{(1)} = S_{(2,1)} + S_{(3)}
    \end{align*}
    を直接計算して示せ。
\end{itembox}

\newpage

\section{Schur多項式の復習}

\begin{itembox}[l]{定義}
    $T \in \text{Tab}^{[m]}(\lambda)$に対して、$x_1, x_2, \ldots, x_m$を変数とする単項式$x^T$を、
    \begin{align*}
        x^T &= \prod_{i \in T\text{(の箱)}} x_i\\
        &= \prod_{i=1}^{m} x_i^{(\text{$T$における$i$の出現回数})}
    \end{align*}
    と定義する。
    これは明らかに積を保存する。すなわち、
    $x^{T_1 \cdot T_2} = x^{T_1}x^{T_2}$
    である。\\
    写像$R_{[m]} \to \Z[x_1, x_2, \ldots, x_m]$を
    \begin{align*}
        \sum_{k=1}^{n} a_k T_k \mapsto \sum_{k=1}^{n} a_k x^{T_k}
    \end{align*}
    で定める。(これは環準同型)。$S_{\lambda}$の像
    \begin{align*}
        s_{\lambda}(x_1, x_2, \ldots, x_m) = \sum_{T \in \text{Tab}^{[m]}(\lambda)} x^T
    \end{align*}
    を$\lambda$のSchur多項式という。
\end{itembox}
    $\lambda = (p)$のときは、
    \begin{align*}
        \text{Tab}^{[m]}(\lambda) = \{ \begin{ytableau}k_1 & k_2 & \cdots & k_p \end{ytableau} | 1 \leq k_1 \leq k_2 \leq \cdots \leq k_p \leq m\}
    \end{align*}
    より、
    \begin{align*}
        s_{(p)}(x_1, \cdots, x_m) = \sum_{T \in \text{Tab}^{[m]}(\lambda)} x^T = \sum_{1 \leq k_1 \leq k_2 \leq \cdots \leq k_p \leq m} x_{k_1}x_{k_2} \cdots x_{k_p}
    \end{align*}
    である。これはすべての$p$次単項式の和であり、完全対称多項式と呼ばれる。これを$h_p(x_1, x_2, \ldots, x_m)$と書くことにする。

    $\lambda = (1^p)$のときは、
    \begin{align*}
        \text{Tab}^{[m]}(\lambda) = \{ \begin{ytableau}k_1 \\ k_2 \\ \vdots \\ k_p \end{ytableau}\mid 1 \leq k_1 < k_2 < \cdots < k_p \leq m\}
    \end{align*}
    より、
    \begin{align*}
        s_{(1^p)}(x_1, \cdots, x_m) = \sum_{T \in \text{Tab}^{[m]}(\lambda)} x^T = \sum_{1 \leq k_1 < k_2 < \cdots < k_p \leq m} x_{k_1}x_{k_2} \cdots x_{k_p}
    \end{align*}
    である。これはすべての異なる変数からなる$p$次単項式の和であり、基本対称多項式と呼ばれる。これを$e_p(x_1, x_2, \ldots, x_m)$と書くことにする。

\begin{itembox}[l]{問題}
    $R_{[2]}$における式
    \begin{align*}
        S_{(2)} \cdot S_{(1)} = S_{(2,1)} + S_{(3)}
    \end{align*}
    に対応する$\Z[x,y]$における恒等式を、$x,y$を用いて具体的に書け。
\end{itembox}




\newpage

\section{Schur多項式}

次のbumpingに関する命題も思い出しておく。
\begin{itembox}[l]{命題1.2}
    \begin{enumerate}
        \item $T \in \text{Tab}(\lambda)$および、$x_1 \leq x_2 \leq \cdots \leq x_p$に対して、
        $U = T \leftarrow x_1 \leftarrow x_2 \leftarrow \cdots \leftarrow x_p$とする。$U \in \text{Tab}(\mu)$とすれば、$\mu / \lambda$はどの箱も同じ列にない。
        \item $T \in \text{Tab}(\lambda)$および、$x_1 > x_2 > \cdots > x_p$に対して、
        $U = T \leftarrow x_1 \leftarrow x_2 \leftarrow \cdots \leftarrow x_p$とする。$U \in \text{Tab}(\mu)$とすれば、$\mu / \lambda$はどの箱も同じ行にない。
        \item $U \in \text{Tab}(\mu)$、$\lambda/\mu$は$p$個の箱からなり、どの箱も同じ列にないとする。このとき、$T \in \text{Tab}(\lambda)$および、$x_1 \leq x_2 \leq \cdots \leq x_p$が一意に存在して、
        $U = T \leftarrow x_1 \leftarrow x_2 \leftarrow \cdots \leftarrow x_p$となる。
        \item $U \in \text{Tab}(\mu)$、$\lambda/\mu$は$p$個の箱からなり、どの箱も同じ行にないとする。このとき、$T \in \text{Tab}(\lambda)$および、$x_1 > x_2 > \cdots > x_p$が一意に存在して、
        $U = T \leftarrow x_1 \leftarrow x_2 \leftarrow \cdots \leftarrow x_p$となる。
    \end{enumerate}
\end{itembox}

証明は1章を参照。

\begin{itembox}[l]{補題}
    \begin{enumerate}[(1)]
        \item $\lambda$をヤング図形とし、$p$を正の整数とする。
        \begin{align*}
            S_\lambda \cdot S_{(p)} = \sum_{\mu} S_{\mu}
        \end{align*}
        ただし、右辺の$\mu$は、$\mu/\lambda$が$p$個の箱からなり、どの箱も同じ列にないようなヤング図形全体を動く。
        \item $\lambda$をヤング図形とし、$p$を正の整数とする。
        \begin{align*}
            S_\lambda \cdot S_{(1^p)} = \sum_{\mu} S_{\mu}
        \end{align*}
        ただし、右辺の$\mu$は、$\mu/\lambda$が$p$個の箱からなり、どの箱も同じ行にないようなヤング図形全体を動く。
    \end{enumerate}
\end{itembox}

\begin{proof}
    \subitem (1) \\
    $S_\lambda \cdot S_{(p)}$の各項は、$T \in \text{Tab}(\lambda)$に$x_1 \leq x_2 \leq \cdots \leq x_p$なる$p$個の数字を並べたヤング図形
    \begin{align*}
        \begin{ytableau}
            x_1 & x_2 & \cdots & x_p
        \end{ytableau} \in \text{Tab}(\mu)
    \end{align*}
    を掛けたものに対応するが、掛けるヤング図形に対応するワードは$x_1 x_2 \cdots x_p$であるから、これは、
    \begin{align*}
        U = T \leftarrow x_1 \leftarrow x_2 \leftarrow \cdots \leftarrow x_p
    \end{align*}
    に等しい。$U \in \text{Tab}(\mu)$とすると、命題1.2より$U$は、$\mu/\lambda$が$p$個の箱からなり、どの箱も同じ列にないようなヤング図形全体を1度ずつ動くので、(1)が成り立つ。

    \subitem (2) \\
    $S_\lambda \cdot S_{(1^p)}$の各項は、$T \in \text{Tab}(\lambda)$に$x_p < x_{p-1} < \cdots < x_1$なる$p$個の数字を並べたヤング図形
    \begin{align*}
        \begin{ytableau}
            x_p \\ \vdots \\ x_2 \\ x_1
        \end{ytableau} \in \text{Tab}(\mu)
    \end{align*}
    を掛けたものに対応するが、掛けるヤング図形に対応するワードは$x_1 x_2 \cdots x_p$であるから、これは、
    \begin{align*}
        U = T \leftarrow x_1 \leftarrow x_2 \leftarrow \cdots \leftarrow x_p
    \end{align*}
    に等しい。$U \in \text{Tab}(\mu)$とすると、命題1.2より$U$は、$\mu/\lambda$が$p$個の箱からなり、どの箱も同じ行にならないようなヤング図形全体を1度ずつ動くので、(1)が成り立つ。
\end{proof}

これより、次の系が得られる

\begin{itembox}[l]{系}
    \begin{align*}
        s_\lambda(x_1, x_2, \ldots, x_m) \cdot h_p(x_1, x_2, \ldots, x_m) &= \sum_{\mu}s_\mu(x_1,x_2,\ldots,x_m) \\
        s_\lambda(x_1, x_2, \ldots, x_m) \cdot e_p(x_1, x_2, \ldots, x_m) &= \sum_{\mu}s_\mu(x_1,x_2,\ldots,x_m) \\
    \end{align*}
    である、ただし、第一式の右辺において$\mu$は$\mu/\lambda$が$p$個の箱からなり、どの箱も同じ列にないようなヤング図形全体を動き、
    第二式の右辺において$\mu$は$\mu/\lambda$が$p$個の箱からなり、どの箱も同じ行にないようなヤング図形全体を動く。
\end{itembox}

後々のために、演習問題3にあたる次の命題を証明しておく。(省略するかも)

\begin{itembox}[l]{命題}
    \begin{enumerate}[(1)]
        \item $\lambda=(\lambda_1, \lambda_2, \ldots, \lambda_k)$をヤング図形とするとき、$\mu / \lambda$が$p$個の箱からなり、どの箱も同じ列にないような歪ヤング図形となるような$\mu$に関する条件は、
        \begin{align*}
            \mu_i \geq \lambda_i \geq \mu_{i+1} \quad (i=1,2,\ldots,k), \quad \mu_{k+2} = 0, \quad \sum_{i} \mu_i = \sum_{i} \lambda_i + p
        \end{align*}
        \item $\lambda=(\lambda_1, \lambda_2, \ldots, \lambda_k)$をヤング図形とするとき、$\mu / \lambda$が$p$個の箱からなり、どの箱も同じ行にないような歪ヤング図形となるような$\mu$に関する条件は、
        \begin{align*}
            \mu_{i} \geq \mu_{i+1},\quad \lambda_i+1 \geq \mu_i \geq \lambda_{i} \quad (i=1,2,\ldots,k), \quad \sum_{i} \mu_i = \sum_{i} \lambda_i + p
        \end{align*}
    \end{enumerate}
    である。
\end{itembox}

\begin{proof}
    \subitem (1) $\To$ \\
    $\mu, \lambda$をヤング図形、$\mu/\lambda$が$p$個の箱からなり、どの箱も同じ列にないとする。
    $\lambda \subset \mu$より、$\mu_i \geq \lambda_i$である。また、$\lambda_i < \mu_{i+1}$なる$i$があるとき、$\mu$における$i+1$行目で一番右側にある箱と、
    その上の箱はどちらも$\mu / \lambda$の箱になる必要があり、これは$\mu / \lambda$のどの2つの箱も同じ列にないという条件に反する。よって、$1 \leq i \leq k$に対して、
    $\mu_i \geq \lambda_i \geq \mu_{i+1}$である。また、$\mu_{k+2} \neq 0$のとき、$\mu$において$k+2$行目と$k+1$行目の一番左側にある箱はいずれも$\lambda$には含まれず、
    これも$\mu / \lambda$のどの2つの箱も同じ列にないという条件に反する。よって、$\mu_{k+2} = 0$である。$\mu / \lambda$は$p$個の箱からなるので、$\sum_{i} \mu_i = \sum_{i} \lambda_i + p$である。

    \subitem (1) $\Longleftarrow$ \\
    逆に、$\lambda, \mu$が条件を満たすとする。$\mu_i \geq \mu_{i+1}$より$\mu$はヤング図形であり、$\mu_i \geq \lambda_i$より$\lambda \subset \mu$である。
    また、$\lambda_i \geq \mu_{i+1}$なので、$i$行目における$\mu / \lambda$の箱は$i+1$行目における$\mu / \lambda$の箱より真に右にある。よって、$\mu / \lambda$の箱はどの2つも同じ列にない。
    また、$\sum_{i} \mu_i = \sum_{i} \lambda_i + p$より、$\mu / \lambda$は$p$個の箱からなる。

    \subitem (2) $\Iff$ \\
    $\mu$がヤング図形であることは、$\mu_i \geq \mu_{i+1}$であることと同値である。また、$\mu / \lambda$が$p$個の箱からなり、どの箱も同じ行にないという条件は、
    どの行にも$\mu / \lambda$の箱は$0$個または$1$個であるということと同値であり、$\lambda_i+1 \geq \mu_i \geq \lambda_{i}$と書き直せる。
    また、$\mu / \lambda$が$p$個の箱からなり、どの箱も同じ行にないという条件は、$\sum_{i} \mu_i = \sum_{i} \lambda_i + p$と書き直せる。
    以上より、(2)が示された。
\end{proof}

\newpage
\section{コストカ数}

\begin{itembox}[l]{定義}
    非負整数の組$\mu = (\mu_1, \mu_2, \ldots, \mu_l)$に対し、タブロー$T$が$\mu_1$個の$1$, $\mu_2$個の$2$, $\ldots$, $\mu_l$個の$l$からなるとき、
    $T$の中身(重み、タイプ)が$\mu$であるという。
    形が$\lambda$で中身が$\mu$であるタブローの個数を$K_{\lambda \mu}$と書く。
    $\mu$も分割であるとき、$K_{\lambda \mu}$はコストカ数と呼ばれる。
\end{itembox}

\begin{itemize}
    \item $\sum_{i} \lambda_i \neq \sum_{i} \mu_i$のとき、$K_{\lambda \mu} = 0$である。
    \item $\lambda$が$n$の分割であるとき、$K_{\lambda (1^n)}$は$\lambda$の標準タブローの個数である。\\
    \sout{したがって$n$番目のカタラン数は$K_{(n,n) (1^{2n})}$である。}
    \item 定義より、$K_{\lambda \mu}$は分割の列
    \begin{align*}
        \lambda^{(1)} \subset \lambda^{(2)} \subset \cdots \subset \lambda^{(l)} = \lambda
    \end{align*}
    であって、$\lambda^{(i+1)}/\lambda^{(i)}$が$\mu_i$個の箱からなり、どの箱も同じ列にないものの個数である。
    ($i$以下の数字が書かれた箱全体はヤング図形であり、それを$\lambda^{(i)}$と書けば明らか)
    \item $K_{\lambda \mu}$は$\mu = (\mu_1, \mu_2, \ldots, \mu_l)$の順序によらないことが第4章で示される。コストカ数を$\mu$が分割である場合に限定するのはおそらくこのためである。
\end{itemize}

\begin{itembox}[l]{補題}
    $\lambda$を分割、$(\mu_1, \mu_2, \ldots, \mu_{l+1})$を非負整数の組とする。
    \begin{align*}
        K_{\lambda (\mu_1, \mu_2, \ldots, \mu_{l+1})} = \sum_{\lambda'} K_{\lambda' (\mu_1, \mu_2, \ldots, \mu_{l})}
    \end{align*}
    である。ここで、$\lambda'$は$\lambda' \subset \lambda$であり、$\lambda / \lambda'$が$\mu_{l+1}$個の箱からなるどの箱も同じ列にない歪ヤング図形であるようなヤング図形全体を動く。
\end{itembox}

\begin{proof}
    形が$\lambda$で中身が$(\mu_1, \mu_2, \ldots, \mu_{l+1})$であるタブロー$T$を決めることは、
    \begin{enumerate}
        \item 中身が$(\mu_1, \mu_2, \ldots, \mu_{l})$であるタブロー$T'$を決める。
        \item $T'$に$\mu_{l+1}$個の$l+1$が書き込まれた箱を追加して、形が$\lambda$のタブロー$T$を作る。
    \end{enumerate}
    ということに他ならない。$T'$の中身が$\mu$であるようなタブローの個数は$K_{\lambda' \mu}$である。$T'$が$\lambda'$の形を持つようなタブローの個数は$K_{\lambda' \mu}$である。
    $\lambda'$が$\lambda' \subset \lambda$であり、$\lambda / \lambda'$が$\mu_{l+1}$個の箱からなり、どの箱も同じ列にないという条件を満たすとき、
    $T'$に$\mu_{l+1}$個の$l+1$が書き込まれた箱を追加して、形が$\lambda$で中身が$(\mu_1, \mu_2, \ldots, \mu_{l+1})$であるタブローを作ることができる。
    また、この作り方はただ1通りである。よって、$K_{\lambda (\mu_1, \mu_2, \ldots, \mu_{l+1})} = \sum_{\lambda'} K_{\lambda' (\mu_1, \mu_2, \ldots, \mu_{l})}$である。
\end{proof}




\begin{itembox}[l]{定理}
    $\mu = (\mu_1, \mu_2, \ldots, \mu_l)$を非負整数の組とする。
    \begin{align*}
        S_{(\mu_1)} \cdot S_{(\mu_2)} \cdot\cdots \cdot S_{(\mu_l)} = \sum_{\lambda} K_{\lambda \mu} S_{\lambda}
    \end{align*}
    である。ここで右辺の$\lambda$は全てのヤング図形を動く。

    言い換えれば、形が$\lambda$のヤング図形$T$が与えられたとき、$T$の中身に関わらず、
    1行$\mu_i$列のタブロー$U_i$を用いて
    \begin{align*}
        T = U_1 \cdot U_2 \cdot \cdots \cdot U_l
    \end{align*}
    と$K_{\lambda \mu}$通りに書ける。
\end{itembox}

\begin{proof}
    % 後半の主張を$l$に関する帰納法で示す。

    % \subitem (1) $l=1$のとき \\
    % 中身が$(\mu_1)$である、すなわち$\mu_1$個の箱からなり、書き込まれた数字がすべて$1$であるタブローは
    % 、形が$\lambda = (\mu_1)$に対するタブロー
    % \begin{align*}
    %     \begin{ytableau}
    %         1 & 1 & \cdots & 1
    %     \end{ytableau}
    % \end{align*}
    % のみである。よって、
    % \begin{align*}
    %     K_{\lambda (\mu_1)} = \begin{cases}
    %         1 & (\lambda = (\mu_1))\\
    %         0 & (\lambda \neq (\mu_1))
    %     \end{cases}
    % \end{align*}
    % である。$T \in \text{Tab}(\lambda)$に対して、$T$の中身に関わらず、$T$を$1$行$\mu_1$列のタブロー$U_1$に書き直す方法は$1$通りであり、
    % $T \not\in \text{Tab}(\lambda)$のときは$U_1$を書くことができないので、0通りである。よって成立する。

    % \subitem (2) $l$で成立するとき、$l+1$で成立することを示す。 \\

    
    \subitem (1) $l=1$のとき \\
    中身が$(\mu_1)$である、すなわち$\mu_1$個の箱からなり、書き込まれた数字がすべて$1$であるタブローは
    $\lambda = (\mu_1)$に対するタブロー
    \begin{align*}
        \begin{ytableau}
            1 & 1 & \cdots & 1
        \end{ytableau}
    \end{align*}
    のみである。よって、
    \begin{align*}
        K_{\lambda (\mu_1)} = \begin{cases}
            1 & (\lambda = (\mu_1))\\
            0 & (\lambda \neq (\mu_1))
        \end{cases}
    \end{align*}
    であるから、
    \begin{align*}
        \sum_{\lambda} K_{\lambda (\mu_1)} S_{\lambda} = S_{(\mu_1)}
    \end{align*}
    である。

    \subitem (2) $l$で成立するとき、$l+1$で成立することを示す。 \\
    \begin{align*}
        S_{(\mu_1)} \cdot S_{(\mu_2)} \cdot \cdots\cdot S_{(\mu_l)} = \sum_{\lambda} K_{\lambda (\mu_1, \mu_2, \ldots, \mu_l)} S_{\lambda}
    \end{align*}
    であると仮定する。
    \begin{align*}
        S_{(\mu_1)}\cdot S_{(\mu_2)}\cdot \cdots\cdot S_{(\mu_l)}\cdot S_{(\mu_{l+1})} &= \sum_{\lambda} K_{\lambda (\mu_1, \mu_2, \ldots, \mu_l)} S_{\lambda} \cdot S_{(\mu_{l+1})} \\
        &= \sum_{\lambda} K_{\lambda (\mu_1, \mu_2, \ldots, \mu_l)} \sum_{\lambda'} S_{\lambda'}
    \end{align*}
    である。ここで、$\mu$は$\mu/\lambda$が$\mu_{l+1}$個の箱からなり、どの箱も同じ列にないようなヤング図形全体を動く。
    和の順序を入れ替えると、
    \begin{align*}
        S_{(\mu_1)}\cdot S_{(\mu_2)}\cdot \cdots\cdot S_{(\mu_l)}\cdot S_{(\mu_{l+1})} &= \sum_{\lambda'} \sum_{\lambda} K_{\lambda (\mu_1, \mu_2, \ldots, \mu_l)} S_{\mu} \\
    \end{align*}
    である。ここで、$\lambda$は$\lambda \subset \lambda'$であり、$\lambda'/\lambda$が$\mu_{l+1}$個の箱からなり、どの箱も同じ列にないようなヤング図形全体を動く。
    ここで、
    \begin{align*}
        K_{\lambda' (\mu_1, \mu_2, \ldots, \mu_{l+1})} = \sum_{\lambda} K_{\lambda (\mu_1, \mu_2, \ldots, \mu_l)}
    \end{align*}
    であるから、
    \begin{align*}
        S_{(\mu_1)}\cdot S_{(\mu_2)}\cdot \cdots\cdot S_{(\mu_l)}\cdot S_{(\mu_{l+1})} &= \sum_{\lambda'} K_{\lambda' (\mu_1, \mu_2, \ldots, \mu_{l+1})} S_{\lambda'} \\
        &= \sum_{\lambda'} K_{\lambda' (\mu_1, \mu_2, \ldots, \mu_{l+1})} S_{\lambda'}
    \end{align*}
    である。よって、$l+1$で成立する。
\end{proof}


\begin{itembox}[l]{問題}
    $R_{[2]}$において、
    \begin{align*}
        S_{(2)} \cdot S_{(2)} \cdot S_{(1)}
    \end{align*}
    を$S_{(5)}, S_{(4,1)}, S_{(3,2)}$の線形結合で表せ。\\
    (注: $R_{[2]}$においては、3行以上のヤング図形$\lambda$に対して$S_{\lambda} = 0$である)
\end{itembox}

\begin{itembox}[l]{系}
    $\mu = (\mu_1, \mu_2, \ldots, \mu_l)$を非負整数の組とする。
    \begin{align*}
        h_{\mu_1}h_{\mu_2}\cdots h_{\mu_l} = \sum_{\lambda} K_{\lambda \mu} s_{\lambda}
    \end{align*}
    である。ここで右辺の$\lambda$は全てのヤング図形を動く。
\end{itembox}

これと同様に、次の命題も成り立つ。

\begin{itembox}[l]{命題}
    $\mu = (\mu_1, \mu_2, \ldots, \mu_l)$を非負整数の組とする。
    \begin{align*}
        e_{\mu_1}e_{\mu_2}\cdots e_{\mu_l} = \sum_{\lambda} K_{\tilde{\lambda} \mu} s_{\lambda}
    \end{align*}
    である。ここで右辺の$\lambda$は全てのヤング図形を動く。
\end{itembox}

証明は同じなので省略する。



\newpage
\section{分割の順序づけとSchur多項式の対称性}


\begin{itembox}[l]{定義}
    $\mu = (\mu_1, \mu_2, \ldots, \mu_l), \lambda = (\lambda_1, \lambda_2, \ldots, \lambda_k)$を分割とする。分割には、$\mu \subset \lambda$の他にも次のような半順序が入る。
    \begin{enumerate}
        \item 辞書式順序: $\mu = \lambda$または、$\mu_i \neq \lambda_i$なる最小の$i$があるとき$\mu_i < \lambda_i$であるとき、$\mu \leq \lambda$と書く。
        \item 支配的順序: 任意の$i$に対して$\mu_1 + \mu_2 + \cdots + \mu_i \leq \lambda_1 + \lambda_2 + \cdots + \lambda_i$が成り立つとき、$\mu \trianglelefteq \lambda$と書く。
    \end{enumerate}
\end{itembox}

辞書式順序は全順序である。また、明らかに
\begin{align*}
    \mu \subset \lambda \To \mu \trianglelefteq \lambda, \quad \mu \trianglelefteq \lambda \To \mu \leq \lambda
\end{align*}
である。


\begin{itembox}[l]{定理}
    $\mu, \lambda$を同じ自然数の分割とする。
    \begin{enumerate}[(1)]
        \item $\mu = \lambda \To K_{\lambda\mu } = 1$
        \item $K_{\lambda\mu } \neq 0 \Iff \mu \trianglelefteq \lambda$
    \end{enumerate}
\end{itembox}

\begin{proof}
    \subitem (1) \\
    数字$i$を$i$行目に入れることで、形、中身がともに$\lambda$であるタブローが存在する。
    これが唯一のタブローであることを$\lambda$の列数$k$に関する帰納法で示す。\\
    $k=1$のとき、形、中身が$\lambda = (\lambda_1)$であるタブローは$\lambda_1$個の$1$からなるタブローのみである。$k>1$のとき、$\lambda = (\lambda_1, \lambda_2, \ldots, \lambda_k)$とする。タブローの数字は下に向かって狭義単調増加であるから、
    $k$行目に入る数字は$k$でなければならない。$k$行目に$k$を入れることで、残りの$k-1$行目には中身、形が$\lambda' = (\lambda_1, \lambda_2, \ldots, \lambda_{k-1})$であるタブローである必要があるが、
    これは帰納法の仮定より$i$行目に$i$を入れる自明なものしか存在しない。よって、$\lambda$の形、中身が$\lambda$であるタブローはただ一つである。

    \subitem (2) $\To$ \\
    対偶「$\mu \not\trianglelefteq \lambda \To K_{\lambda\mu } = 0$」を示す。$\mu \not\trianglelefteq \lambda$とすると、ある$i$が存在して$\mu_1 + \mu_2 + \cdots + \mu_i > \lambda_1 + \lambda_2 + \cdots + \lambda_i$である。
    したがって、$i$以下の数字を$i$行目までにすべて入れることはできない。よって、$i+1$行目以降に$i$以下の数字が入っていることになるが、箱の数字が下に向かって狭義単調増加であるため、このようなタブローは存在しない。
    以上より、$\mu \not\trianglelefteq \lambda \To K_{\lambda\mu } = 0$である。

    \subitem (2) $\Longleftarrow$ \\
    $\mu_l \neq 0$なる最大の$l$についての帰納法を用いる。$l=0$のときは明らか。$l \geq 1$とする。$n$を$\mu, \lambda$の箱の数とする。
    $\lambda$から箱を次のような優先順位で選ぶことを$\mu_l$回繰り返し、選んだ箱を取り除くことで新たなヤング図形$\lambda'$を得る。\\
    \quad 1. 今まで選んだ箱が同じ列にない。\\
    \quad 2. なるべく下の箱を選ぶ。\\
    \quad 3. なるべく右の箱を選ぶ。\\
    $\lambda_1 \geq \mu_1 \geq \mu_l$より、1を満たすように$\mu_l$個の箱を選ぶことは可能である。
    $\lambda'$の中身は$\mu' = (\mu_1, \mu_2, \cdots, \mu_{l-1},0)$である。
    ここでもし$\mu' \trianglelefteq \lambda'$であることが示されれば、中身が$\mu'$、形が$\lambda'$のタブローが存在するので、それに$\lambda/\lambda'$にあたる箱に$l$を書き込むことで、
    中身が$\mu$、形が$\lambda$のタブローを得ることができる。よって$\mu' \trianglelefteq \lambda'$を示そう。
    $\lambda_1 \geq \mu_l, \lambda_l \leq \mu_l$より、$\lambda_j \geq \mu_l$を満たす最大の$j$がとれる。このとき、
    \begin{align*}
        \lambda' = (\lambda_1, \lambda_2, \cdots, \lambda_{j-1}, \lambda_{j} + \lambda_{j+1} - \mu_l, \lambda_{j+2}, \lambda_{j+3}, \cdots, \lambda_l,0)
    \end{align*}
    である。$i < j$に対して$\lambda'_1 + \lambda'_2 + \cdots +  \lambda'_i \leq \mu'_1 + \mu'_2 + \cdots +  \mu'_i$は$\lambda \trianglelefteq \mu$より明らか。
    $i \geq l$については$\mu_i' = 0$なので、$j < i \leq l$について考えればよい。
    \begin{align*}
        & (\lambda_1' + \lambda_2' + \cdots + \lambda_j') - (\mu_1' + \mu_2' + \cdots + \mu_j') \\ 
        &= (\mu_{j+1}' + \mu_{j+2}' + \cdots + \mu_l') - (\lambda_{j+1}' + \lambda_{j+2}' + \cdots + \lambda_l') \\
        &= (\mu_{j+1} + \mu_{j+2} + \cdots + \mu_{l-1}) - (\lambda_{j+2} + \lambda_{j+3} + \cdots + \lambda_l) \\
        &= (\mu_{j+1} - \mu_{l}) + \{(\mu_{j+2} + \cdots + \mu_{l}) - (\lambda_{j+2} + \lambda_{j+3} + \cdots + \lambda_l)\} \\
        &\geq 0
    \end{align*}
    であるから、
    \begin{align*}
        \mu' \trianglelefteq \lambda'
    \end{align*}
    なので、示された。
\end{proof}

なお、(2)は$\mu$が分割でないときは成り立たないことに注意せよ。例えば$\lambda = (2,2), \mu = (1,3)$


\begin{itembox}[l]{定理}
    任意のヤング図形に対するSchur多項式は対称多項式である。
\end{itembox}

\begin{proof}

    $n$の分割すべてを辞書式順序で並べ、
    \begin{align*}
        \lambda^{(1)} \leq \lambda^{(2)} \leq \cdots \leq \lambda^{(m)}
    \end{align*}
    とする。また、行列
    \begin{align*}
        (K_{\lambda^{(i)} \lambda^{(j)}})_{{1 \leq i,j \leq m}}
    \end{align*}
    を考える。$i \leq j$でないとき、$K_{\lambda^{(i)} \lambda^{(j)}} = 0$であり、さらに任意の$i$に対して$K_{\lambda^{(i)} \lambda^{(i)}} = 1$であるから、
    これは下三角行列であり、対角成分はすべて$1$である。よって$K$は正則である。また、

    \begin{align*}
        h_{\mu_1}h_{\mu_2}\cdots h_{\mu_l} = \sum_{\lambda} K_{\lambda \mu} s_{\lambda}
    \end{align*}
    より、
    \begin{align*}
        (s_{\lambda^{(1)}}, s_{\lambda^{(2)}}, \ldots, s_{\lambda^{(m)}}) \pmat{K_{\lambda^{(1)} \lambda^{(j)}} \\ K_{\lambda^{(2)} \lambda^{(j)}} \\ \vdots \\ K_{\lambda^{(m)} \lambda^{(j)}}} = h_{\lambda^{(j)}_1} h_{\lambda^{(j)}_2} \cdots h_{\lambda^{(j)}_m}
    \end{align*}
    なので、

    \begin{align*}
        &(s_{\lambda^{(1)}}, s_{\lambda^{(2)}}, \ldots, s_{\lambda^{(m)}}) 
        % \pmat{
        %     K_{\lambda^{(1)} \lambda^{(1)}} & K_{\lambda^{(1)} \lambda^{(2)}} & \cdots & K_{\lambda^{(1)} \lambda^{(m)}} \\
        %     0 & K_{\lambda^{(2)} \lambda^{(2)}} & \cdots & K_{\lambda^{(2)} \lambda^{(m)}} \\
        %     \vdots & \vdots & \ddots & \vdots \\
        %     0 & 0 & \cdots & K_{\lambda^{(m)} \lambda^{(m)}}
        % } \\&\qquad
        K
        = (h_{\lambda^{(1)}_1} h_{\lambda^{(1)}_2} \cdots h_{\lambda^{(1)}_m}, h_{\lambda^{(2)}_1} h_{\lambda^{(2)}_2} \cdots h_{\lambda^{(2)}_m}, \ldots, h_{\lambda^{(m)}_1} h_{\lambda^{(m)}_2} \cdots h_{\lambda^{(m)}_m})
    \end{align*}
    であるから、
    \begin{align*}
        (s_{\lambda^{(1)}}, s_{\lambda^{(2)}}, \ldots, s_{\lambda^{(m)}}) = (h_{\lambda^{(1)}_1} h_{\lambda^{(1)}_2} \cdots h_{\lambda^{(1)}_m}, h_{\lambda^{(2)}_1} h_{\lambda^{(2)}_2} \cdots h_{\lambda^{(2)}_m}, \ldots, h_{\lambda^{(m)}_1} h_{\lambda^{(m)}_2} \cdots h_{\lambda^{(m)}_m})K^{-1}
    \end{align*}
    より、$s_{\lambda^{(i)}}$は完全対称多項式のみで表されるので、$s_{\lambda^{(i)}}$は対称多項式である。
\end{proof}

$K$の行列式が$1$なので、Schur多項式は完全対称多項式の整数係数の線形結合で表されることもわかる。


% 具体的にやってみよう。$m=3, n=2$のとき、ヤング図形は
% \begin{align*}
%     \lambda^{(1)} &= (1,1), \quad \lambda^{(2)} = (2)
% \end{align*}
% であり、
% \begin{align*}
%     h_{1}h_{1} &= K_{(1,1)(1,1)}s_{(1,1)} + K_{(1,1)(2)}s_{(2)} = s_{(1,1)} + s_{(2)} \\
%     h_{2} &= K_{(2)(1,1)}s_{(1,1)} + K_{(2)(2)}s_{(2)} = s_{(2)}
% \end{align*}
% であるから、
% \begin{align*}
%     (s_{(1,1)}, s_{(2)}) \pmat{1 & 0 \\ 1 & 1} &= (h_1h_1, h_2) \\
%     \therefore (s_{(1,1)}, s_{(2)}) &= (h_1h_1, h_2) \pmat{1 & 0 \\ -1 & 1} = (h_1h_1 - h_2, h_2)
% \end{align*}
% よって、
% \begin{align*}
%     s_{(1,1)} &= h_1h_1 - h_2 = (x+y)^2 - (x^2 + xy + y^2) = xy \\
%     s_{(2)} &= h_2 = x^2 + xy + y^2
% \end{align*}

\newpage
\section{問題の解答}

\begin{itembox}[l]{問題}
    $R_{[2]}$において、
    \begin{align*}
        S_{(2)} \cdot S_{(1)} = S_{(2,1)} + S_{(3)}
    \end{align*}
    を計算して示せ。
\end{itembox}

\begin{align*}
    S_{(2)} &= \ytableaushort{11} + \ytableaushort{12} + \ytableaushort{22} \\
    S_{(1)} &= \ytableaushort{1} + \ytableaushort{2} \\
    S_{(2,1)} &= \ytableaushort{11,2} + \ytableaushort{12,2} \\
    S_{(3)} &= \ytableaushort{111} + \ytableaushort{112} + \ytableaushort{122} + \ytableaushort{222}
\end{align*}
である。
\begin{align*}
    S_{(2)} \cdot S_{(1)} &= \(\ytableaushort{11} + \ytableaushort{12} + \ytableaushort{22}\) \cdot \(\ytableaushort{1} + \ytableaushort{2}\) \\
    &= \ytableaushort{111} + \ytableaushort{11,2} + \ytableaushort{12,2} + \ytableaushort{112} + \ytableaushort{122} + \ytableaushort{222} \\
    &= S_{(3)} + S_{(2,1)}
\end{align*}

\begin{itembox}[l]{問題}
    $R_{[2]}$における式
    \begin{align*}
        S_{(2)} \cdot S_{(1)} = S_{(2,1)} + S_{(3)}
    \end{align*}
    に対応する$\Z[x,y]$における恒等式を、$x,y$を用いて具体的に書け。
\end{itembox}

\begin{align*}
    s_{(2)} &= x^2 + xy + y^2 \\
    s_{(1)} &= x + y \\
    s_{(2,1)} &= x^2y + xy^2 \\
    s_{(3)} &= x^3 + x^2y + xy^2 + y^3
\end{align*}

であるから、

\begin{align*}
    (x^2 + xy + y^2)(x+y) &= (x^2y + xy^2) + (x^3 + x^2y + xy^2 + y^3) \\
\end{align*}

\begin{itembox}[l]{問題}
    $R_{[2]}$において、
    \begin{align*}
        S_{(2)} \cdot S_{(2)} \cdot S_{(1)}
    \end{align*}
    を$S_{(5)}, S_{(4,1)}, S_{(3,2)}$の線形結合で表せ。\\
    (注: $R_{[2]}$においては、3行以上のヤング図形$\lambda$に対して$S_{\lambda} = 0$である)
\end{itembox}


\begin{align*}
    S_{(2)} \cdot S_{(2)} \cdot S_{(1)} &= \sum_{\lambda} K_{\lambda (2,2,1)} S_{\lambda}
\end{align*}

である。$K_{\lambda (2,2,1)} \neq 0$となるためには、少なくとも$\lambda$が$5$の分割であり、2行以下である必要がある。よって、$S_{(5)}$の係数は$1$である。$K_{\lambda (2,2,1)} \neq 0$となるためには、少なくとも$\lambda$が$4,1$の分割であり、2行以下である必要がある。よって、$S_{(4,1)}$の係数は$1$である。$K_{\lambda (2,2,1)} \neq 0$となるためには、少なくとも$\lambda$が$5$の分割であり、2行以下である必要がある。よって、
\begin{align*}
    \lambda = (5), (4,1), (3,2)
\end{align*}
について考えれば十分である。

\begin{align*}
    K_{(5) (2,2,1)} &= \sharp \{
        \ytableaushort{11223}
    \} = 1 \\
    K_{(4,1) (2,2,1)} &= \sharp \{
        \ytableaushort{1122,3}, \ytableaushort{1123,2}
    \} = 2 \\
    K_{(3,2) (2,2,1)} &= \sharp \{
        \ytableaushort{112,23}, \ytableaushort{113,22}
    \} = 2
\end{align*}
であるから、

\begin{align*}
    S_{(2)} \cdot S_{(2)} \cdot S_{(1)} = S_{(5)} + 2S_{(4,1)} + 2S_{(3,2)}
\end{align*}

である。

おまけ: 直接計算すると、
\begin{align*}
    S_{(2)} \cdot S_{(2)} \cdot S_{(1)} &= \(\ytableaushort{11} + \ytableaushort{12} + \ytableaushort{22}\) \cdot \(\ytableaushort{11} + \ytableaushort{12} + \ytableaushort{22}\) \cdot \(\ytableaushort{1} + \ytableaushort{2}\) \\
    &= \( \splitfrac{\ytableaushort{1111} + \ytableaushort{1112} + \ytableaushort{1122} }{\splitfrac{+\ \ytableaushort{111,2} + \ytableaushort{112,2}+ \ytableaushort{1122}}{\ + \ytableaushort{11,22} + \ytableaushort{122,2} + \ytableaushort{2222}}} \) \cdot \(\ytableaushort{1} + \ytableaushort{2}\) \\
    &= \ytableaushort{11111} + \ytableaushort{11112} + \ytableaushort{1111,2} +  \ytableaushort{11122} \\
    &+ \ytableaushort{1112,2} + \ytableaushort{11222} + \ytableaushort{1111,2}+ \ytableaushort{111,22} \\
    &+ \ytableaushort{111,22} + \ytableaushort{1122,2}+ \ytableaushort{1112,2}+ \ytableaushort{11222} \\
    &+ \ytableaushort{111,22} + \ytableaushort{112,22} + \ytableaushort{112,22} +  \ytableaushort{1222,2} \\
    &+ \ytableaushort{1222,2} + \ytableaushort{22222} \\
    &= \( \splitfrac{\ytableaushort{11111} + \ytableaushort{11112} + \ytableaushort{11122}}{+ \ytableaushort{11222}+ \ytableaushort{12222}+ \ytableaushort{22222}} \) \\
    &+ 2\( \ytableaushort{1111,2} + \ytableaushort{1112,2} + \ytableaushort{1122,2} + \ytableaushort{1222,2} \) \\
    &+ 2\( \ytableaushort{111,22} + \ytableaushort{112,22} \) \\
    &= S_{(5)} + 2S_{(4,1)} + 2S_{(3,2)}
\end{align*}
となる。(二度とやらない)





\end{document}